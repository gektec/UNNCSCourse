\documentclass{article}
%************Preamble*************

\usepackage{amsmath}
\usepackage{amssymb}

\newcommand{\ds}{\displaystyle} 
% User-defined new command \ds for replacing a longer one \displaystyle.
\newcommand{\crt}{\sqrt[3]} 
% User-defined new command \crt for Cube RooT.

\begin{document}
%************Content**************

{\Large Lecture 2 Notes}\\[1cm] 

\textbf{Inline math vs. Display math:} \\ % \textbf{} is used for boldface font

Differences between inline math $\sum_{k=1}^n$, $\int_{10}^{20} f(x)\, dx$, $\frac{1}{x}$
and display math
\[
\sum_{k=1}^n
\]

\[
\int_{10}^{20} f(x)\, dx
\]

\[
\frac{1}{x}
\]

\vspace{1cm}
%************New Topic **************
\textbf{\textbackslash displaystyle command:}\\

Compare the differences between $f(x)=\frac{1}{1+x^2}$ and $f(x)=\displaystyle\frac{1}{1+x^2}$

\vspace{1cm}
%************New Topic **************
\textbf{User-defined new command:}\\

% Example 1
Same outputs for $\ds \frac{1}{1+x^2}$ and $\displaystyle \frac{1}{1+x^2}$.\\

% Example 2
Same outputs for $\crt{x}$ and $\sqrt[3]{x}$.\\


\vspace{1cm}
%************New Topic **************
\textbf{Fitting bracket size:}\\

\[
(\frac{1}{1+x^2})^3 % not a proper bracket size 
\]  

\[
\left(\frac{1}{1+x^2}\right)^3 % a good one
\]

\[
\lim_{x\to 0} \left( \frac{\sin x}{x} \right)^3 = 1 % a good one
\]

\[
\left| 
\frac{1-i}{1+i} % a good one
\right|
\]

\vspace{1cm}
%************New Topic **************
\textbf{Equation \& Eqnarray:}\\ 

\begin{equation}% Equation Example 1
x + \sin x = x^2
\end{equation}
\begin{equation}% Equation Example 2
\frac{dy}{dx} = \tan x
\end{equation}
\begin{equation}% Equation Example 3
\sqrt[3]{x} + \log_2 x = \pi
\end{equation}


\begin{eqnarray} % Eqnarray Example 1
x + \sin x &=& x^2 \\
\frac{dy}{dx} &=& \tan x \\
\sqrt[3]{x} + \log_2 x &=& \pi
\end{eqnarray}

\begin{eqnarray} % Eqnarray Example 2
(1+x)^4 &=& (1+x)^2\cdot (1+x)^2 \\
&=& (1+2x+x^2)\cdot (1+2x+x^2)\\
&=& 1 + 4x + 6x^2 + 4x^3 + x^4
\end{eqnarray}

\begin{eqnarray} % Eqnarray Example 3
&&(1+x)^4 \\
&=& (1+x)^2\cdot (1+x)^2 \\
&=& (1+2x+x^2)\cdot (1+2x+x^2)\\
&=& \cdots
\end{eqnarray}

\vspace{1cm}
%************New Topic **************
\textbf{Suppressing Equation Numbers:}\\

\begin{equation*}
\sin 2\theta = 2 \sin\theta\cos\theta
\end{equation*}


\begin{eqnarray*} % Eqnarray Example 4
(1+x)^4 &=& (1+x)^2\cdot (1+x)^2 \\
&=& (1+2x+x^2)\cdot (1+2x+x^2)\\
&=& 1 + 4x + 6x^2 + 4x^3 + x^4
\end{eqnarray*}

\begin{eqnarray} % Eqnarray Example 5
(1+x)^4 &=& (1+x)^2\cdot (1+x)^2 \nonumber \\
&=& (1+2x+x^2)\cdot (1+2x+x^2) \\
&=& 1 + 4x + 6x^2 + 4x^3 + x^4 \nonumber
\end{eqnarray}

\newpage % starting a new page

\textbf{Arrays:}\\

\[ % math environment needed
\begin{array}{rl} % Array Example 1
12 & 33 \\
5 & 7 \\
29 & 11
\end{array}
\]

$$
\begin{array}{ccc} % Array Example 2
x^{10} & \cos x & \ln x\\
10x^9 & -\sin x &\frac{1}{x}
\end{array}
$$

\vspace{1cm}
%************New Topic **************
\textbf{Matrix:}\\

$
\left(
\begin{array}{rl} % Matrix Example 1
12 & 33 \\
5 & 7 \\
29 & 11
\end{array}
\right)
$
\hspace{8mm}
$
\left[
\begin{array}{cc} % Matrix Example 2
12 & 33 \\
5 & 7 \\
29 & 11
\end{array}
\right]
$
\newline

\[
A=
\left[
\begin{array}{cc} % Matrix Example 3
1 & 2 \\
3 & 4
\end{array}
\right]
\Rightarrow
\det(A)=1\times 4-2\times 3 = -2
\]


\vspace{1cm}
%************New Topic **************
\textbf{Piecewise function: }\\

\[
|x| = 
\left\{ % left curly bracket
\begin{array}{lcc} %Piecewise function Example 1
x & ; & x>0\\
0 & ; & x=0\\
-x & ; & x<0
\end{array}
\right. % pairing with \left
\]

$$
|x| = 
\left\{ 
\begin{array}{rc} %Piecewise function Example 2
x; & x\ge 0\\
-x; & x<0 
\end{array}
\right.
$$


\newpage
%************New Topic **************
\textbf{Tabular:} \\

\begin{tabular}{|c|c|}  % Table Example 1
\hline
True & False \\
\hline
1 & 0 \\
\hline
\end{tabular}

\vspace{5mm}

\begin{tabular}{c|rr} % Table Example 2
$f(x)$ &  $f'(x)$  &  $f''(x)$\\[1ex]
\hline
$x^2$ &  $2x$   &  2\\
\hline
$\sin x$ & $\cos x$ & $-\sin x$
\end{tabular}


\vspace{1cm}
\textbf{Table:} \\

\begin{table}[h]  % this table goes here
\centering % for centering the table
\begin{tabular}{|c|c|c|c|} % Table Example 3
\hline
Name & Age & Gender & Group \\
\hline
Alice & 18 & F & C21\\
\hline
Bob & 20 & M & A29\\
\hline
David & 19 & M & C25\\
\hline
\end{tabular}
\caption{Student list} % adding table discription
\end{table}









%************End**************
\end{document}